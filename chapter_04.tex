\chapter{}
\begin{enumerate}
\item Having dwelt with a teacher during the fourth part of (a man's) life, a Brahmana shall live during the second quarter (of his existence) in his house, after he has wedded a wife.
\item A Brahmana must seek a means of subsistence which either causes no, or at least little pain (to others), and live (by that) except in times of distress.
\item For the purpose of gaining bare subsistence, let him accumulate property by (following those) irreproachable occupations (which are prescribed for) his (caste), without (unduly) fatiguing his body.
\item He may subsist by Rita (truth), and Amrita (ambrosia), or by Mrita (death) and by Pramrita (what causes many deaths); or even by (the mode) called Satyanrita (a mixture of truth and falsehood), but never by Svavritti (a dog's mode of life).
\item By Rita shall be understood the gleaning of corn; by Amrita, what is given unasked; by Mrita, food obtained by begging and agriculture is declared to be Pramrita.
\item But trade and (money-lending) are Satyanrita, even by that one may subsist. Service is called Svavritti; therefore one should avoid it.
\item He may either possess enough to fill a granary, or a store filling a grain-jar; or he may collect what suffices for three days, or make no provision for the morrow.
\item Moreover, among these four Brahmana householders, each later-(named) must be considered more distinguished, and through his virtue to have conquered the world more completely.
\item One of these follows six occupations, another subsists by three, one by two, but the fourth lives by the Brahmasattra.
\item He who maintains himself by picking up grains and ears of corn, must be always intent on (the performance of) the Agnihotra, and constantly offer those Ishtis only, which are prescribed for the days of the conjunction and opposition (of the moon), and for the solstices.
\item Let him never, for the sake of subsistence, follow the ways of the world; let him live the pure, straightforward, honest life of a Brahmana.
\item He who desires happiness must strive after a perfectly contented disposition and control himself; for happiness has contentment for its root, the root of unhappiness is the contrary (disposition).
\item A Brahmana, who is a Snataka and subsists by one of the (above-mentioned) modes of life, must discharge the (following) duties which secure heavenly bliss, long life, and fame.
\item Let him, untired, perform daily the rites prescribed for him in the Veda; for he who performs those according to his ability, attains to the highest state.
\item Whether he be rich or even in distress, let him not seek wealth through pursuits to which men cleave, nor by forbidden occupations, nor (let him accept presents) from any (giver whosoever he may be).
\item Let him not, out of desire (for enjoyments), attach himself to any sensual pleasures, and let him carefully obviate an excessive attachment to them, by (reflecting on their worthlessness in) his heart.
\item Let him avoid all (means of acquiring) wealth which impede the study of the Veda; (let him maintain himself) anyhow, but study, because that (devotion to the Veda-study secures) the realisation of his aims.
\item Let him walk here (on earth), bringing his dress, speech, and thoughts to a conformity with his age, his occupation, his wealth, his sacred learning, and his race.
\item Let him daily pore over those Institutes of science which soon give increase of wisdom, those which teach the acquisition of wealth, those which are beneficial (for other worldly concerns), and likewise over the Nigamas which explain the Veda.
\item For the more a man completely studies the Institutes of science, the more he fully understands (them), and his great learning shines brightly.
\item Let him never, if he is able (to perform them), neglect the sacrifices to the sages, to the gods, to the Bhutas, to men, and to the manes.
\item Some men who know the ordinances for sacrificial rites, always offer these great sacrifices in their organs (of sensation), without any (external) effort.
\item Knowing that the (performance of the) sacrifice in their speech and their breath yields imperishable (rewards), some always offer their breath in their speech, and their speech in their breath.
\item Other Brahmanas, seeing with the eye of knowledge that the performance of those rites has knowledge for its root, always perform them through knowledge alone.
\item A Brahmana shall always offer the Agnihotra at the beginning or at the end of the day and of the night, and the Darsa and Paurnamasa (Ishtis) at the end of each half-month,
\item When the old grain has been consumed the (Agrayana) Ishti with new grain, at the end of the (three) seasons the (Katurmasya-) sacrifices, at the solstices an animal (sacrifice), at the end of the year Soma-offerings.
\item A Brahmana, who keeps sacred fires, shall, if he desires to live long, not eat new grain or meat, without having offered the (Agrayana) Ishti with new grain and an animal-(sacrifice).
\item For his fires, not being worshipped by offerings of new grain and of an animal, seek to devour his vital spirits, (because they are) greedy for new grain and flesh.
\item No guest must stay in his house without being honoured, according to his ability, with a seat, food, a couch, water, or roots and fruits.
\item Let him not honour, even by a greeting, heretics, men who follow forbidden occupations, men who live like cats, rogues, logicians, (arguing against the Veda,) and those who live like herons.
\item Those who have become Snatakas after studying the Veda, or after completing their vows, (and) householders, who are Srotriyas, one must worship by (gifts of food) sacred to gods and manes, but one must avoid those who are different.
\item A householder must give (as much food) as he is able (to spare) to those who do not cook for themselves, and to all beings one must distribute (food) without detriment (to one's own interest).
\item A Snataka who pines with hunger, may beg wealth of a king, of one for whom he sacrifices, and of a pupil, but not of others; that is a settled rule.
\item A Snataka who is able (to procure food) shall never waste himself with hunger, nor shall he wear old or dirty clothes, if he possesses property.
\item Keeping his hair, nails, and beard clipped, subduing his passions by austerities, wearing white garments and (keeping himself) pure, he shall be always engaged in studying the Veda and (such acts as are) conducive to his welfare.
\item He shall carry a staff of bamboo, a pot full of water, a sacred string, a bundle of Kusa grass, and (wear) two bright golden ear-rings.
\item Let him never look at the sun, when he sets or rises, is eclipsed or reflected in water, or stands in the middle of the sky.
\item Let him not step over a rope to which a calf is tied, let him not run when it rains, and let him not look at his own image in water; that is a settled rule.
\item Let him pass by (a mound of) earth, a cow, an idol, a Brahmana, clarified butter, honey, a crossway, and well-known trees, turning his right hand towards them.
\item Let him, though mad with desire, not approach his wife when her courses appear; nor let him sleep with her in the same bed.
\item For the wisdom, the energy, the strength, the sight, and the vitality of a man who approaches a woman covered with menstrual excretions, utterly perish.
\end{enumerate}
