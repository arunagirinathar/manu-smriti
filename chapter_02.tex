\chapter{}
\begin{enumerate}
\item Learn that sacred law which is followed by men learned (in the Veda) and assented to in their hearts by the virtuous, who are ever exempt from hatred and inordinate affection.
\item To act solely from a desire for rewards is not laudable, yet an exemption from that desire is not (to be found) in this (world): for on (that) desire is grounded the study of the Veda and the performance of the actions, prescribed by the Veda.
\item The desire (for rewards), indeed, has its root in the conception that an act can yield them, and in consequence of (that) conception sacrifices are performed; vows and the laws prescribing restraints are all stated to be kept through the idea that they will bear fruit.
\item Not a single act here (below) appears ever to be done by a man free from desire; for whatever (man) does, it is (the result of) the impulse of desire.
\item He who persists in discharging these (prescribed duties) in the right manner, reaches the deathless state and even in this (life) obtains (the fulfilment of) all the desires that he may have conceived.
\item The whole Veda is the (first) source of the sacred law, next the tradition and the virtuous conduct of those who know the (Veda further), also the customs of holy men, and (finally) self-satisfaction.
\item Whatever law has been ordained for any (person) by Manu, that has been fully declared in the Veda: for that (sage was) omniscient.
\item But a learned man after fully scrutinising all this with the eye of knowledge, should, in accordance with the authority of the revealed texts, be intent on (the performance of) his duties.
\item For that man who obeys the law prescribed in the revealed texts and in the sacred tradition, gains fame in this (world) and after death unsurpassable bliss.
\item But by Sruti (revelation) is meant the Veda, and by Smriti (tradition) the Institutes of the sacred law: those two must not be called into question in any matter, since from those two the sacred law shone forth.
\item Every twice-born man, who, relying on the Institutes of dialectics, treats with contempt those two sources (of the law), must be cast out by the virtuous, as an atheist and a scorner of the Veda.
\item The Veda, the sacred tradition, the customs of virtuous men, and one's own pleasure, they declare to be visibly the fourfold means of defining the sacred law.
\item The knowledge of the sacred law is prescribed for those who are not given to the acquisition of wealth and to the gratification of their desires; to those who seek the knowledge of the sacred law the supreme authority is the revelation (Sruti).
\item But when two sacred texts (Sruti) are conflicting, both are held to be law; for both are pronounced by the wise (to be) valid law.
\item (Thus) the (Agnihotra) sacrifice may be (optionally) performed, at any time after the sun has risen, before he has risen, or when neither sun nor stars are visible; that (is declared) by Vedic texts.
\item Know that he for whom (the performance of) the ceremonies beginning with the rite of impregnation (Garbhadhana) and ending with the funeral rite (Antyeshti) is prescribed, while sacred formulas are being recited, is entitled (to study) these Institutes, but no other man whatsoever.
\item That land, created by the gods, which lies between the two divine rivers Sarasvati and Drishadvati, the (sages) call Brahmavarta.
\item The custom handed down in regular succession (since time immemorial) among the (four chief) castes (varna) and the mixed (races) of that country, is called the conduct of virtuous men.
\item The plain of the Kurus, the (country of the) Matsyas, Pankalas, and Surasenakas, these (form), indeed, the country of the Brahmarshis (Brahmanical sages, which ranks) immediately after Brahmavarta.
\item From a Brahmana, born in that country, let all men on earth learn their several usages.
\item That (country) which (lies) between the Himavat and the Vindhya (mountains) to the east of Prayaga and to the west of Vinasana (the place where the river Sarasvati disappears) is called Madhyadesa (the central region).
\item But (the tract) between those two mountains (just mentioned), which (extends) as far as the eastern and the western oceans, the wise call Aryavarta (the country of the Aryans).
\item That land where the black antelope naturally roams, one must know to be fit for the performance of sacrifices; (the tract) different from that (is) the country of the Mlekkhas (barbarians).
\item Let twice-born men seek to dwell in those (above-mentioned countries); but a Sudra, distressed for subsistence, may reside anywhere.
\item Thus has the origin of the sacred law been succinctly described to you and the origin of this universe; learn (now) the duties of the castes (varna).
\item With holy rites, prescribed by the Veda, must the ceremony on conception and other sacraments be performed for twice-born men, which sanctify the body and purify (from sin) in this (life) and after death.
\item By burnt oblations during (the mother's) pregnancy, by the Gatakarman (the ceremony after birth), the Kauda (tonsure), and the Maungibandhana (the tying of the sacred girdle of Munga grass) is the taint, derived from both parents, removed from twice-born men.
\item By the study of the Veda, by vows, by burnt oblations, by (the recitation of) sacred texts, by the (acquisition of the) threefold sacred science, by offering (to the gods, Rishis, and manes), by (the procreation of) sons, by the great sacrifices, and by (Srauta) rites this (human) body is made fit for (union with) Brahman.
\item Before the navel-string is cut, the Gatakarman (birth-rite) must be performed for a male (child); and while sacred formulas are being recited, he must be fed with gold, honey, and butter.
\item But let (the father perform or) cause to be performed the Namadheya (the rite of naming the child), on the tenth or twelfth (day after birth), or on a lucky lunar day, in a lucky muhurta, under an auspicious constellation.
\item Let (the first part of) a Brahmana's name (denote something) auspicious, a Kshatriya's be connected with power, and a Vaisya's with wealth, but a Sudra's (express something) contemptible.
\item (The second part of) a Brahmana's (name) shall be (a word) implying happiness, of a Kshatriya's (a word) implying protection, of a Vaisya's (a term) expressive of thriving, and of a Sudra's (an expression) denoting service.
\item The names of women should be easy to pronounce, not imply anything dreadful, possess a plain meaning, be pleasing and auspicious, end in long vowels, and contain a word of benediction.
\item In the fourth month the Nishkramana (the first leaving of the house) of the child should be performed, in the sixth month the Annaprasana (first feeding with rice), and optionally (any other) auspicious ceremony required by (the custom of) the family.
\item According to the teaching of the revealed texts, the Kudakarman (tonsure) must be performed, for the sake of spiritual merit, by all twice-born men in the first or third year.
\item In the eighth year after conception, one should perform the initiation (upanayana) of a Brahmana, in the eleventh after conception (that) of a Kshatriya, but in the twelfth that of a Vaisya.
\item (The initiation) of a Brahmana who desires proficiency in sacred learning should take place in the fifth (year after conception), (that) of a Kshatriya who wishes to become powerful in the sixth, (and that) of a Vaisya who longs for (success in his) business in the eighth.
\item The (time for the) Savitri (initiation) of a Brahmana does not pass until the completion of the sixteenth year (after conception), of a Kshatriya until the completion of the twenty-second, and of a Vaisya until the completion of the twenty-fourth.
\item After those (periods men of) these three (castes) who have not received the sacrament at the proper time, become Vratyas (outcasts), excluded from the Savitri (initiation) and despised by the Aryans.
\item With such men, if they have not been purified according to the rule, let no Brahmana ever, even in times of distress, form a connexion either through the Veda or by marriage.
\item Let students, according to the order (of their castes), wear (as upper dresses) the skins of black antelopes, spotted deer, and he-goats, and (lower garments) made of hemp, flax or wool.
\item The girdle of a Brahmana shall consist of a of a triple cord of Munga grass, smooth and soft; (that) of a Kshatriya, of a bowstring, made of Murva fibres; (that) of a Vaisya, of hempen threads.
\item If Munga grass (and so forth) be not procurable, (the girdles) may be made of Kusa, Asmantaka, and Balbaga (fibres), with a single threefold knot, or with three or five (knots according to the custom of the family).
\item The sacrificial string of a Brahmana shall be made of cotton, (shall be) twisted to the right, (and consist) of three threads, that of a Kshatriya of hempen threads, (and) that of a Vaisya of woollen threads.
\item A Brahmana shall (carry), according to the sacred law, a staff of Bilva or Palasa; a Kshatriya, of Vata or Khadira; (and) a Vaisya, of Pilu or Udumbara.
\item The staff of a Brahmana shall be made of such length as to reach the end of his hair; that of a Kshatriya, to reach his forehead; (and) that of a Vaisya, to reach (the tip of his) nose.
\item Let all the staves be straight, without a blemish, handsome to look at, not likely to terrify men, with their bark perfect, unhurt by fire.
\item Having taken a staff according to his choice, having worshipped the sun and walked round the fire, turning his right hand towards it, (the student) should beg alms according to the prescribed rule.
\item An initiated Brahmana should beg, beginning (his request with the word) lady (bhavati); a Kshatriya, placing (the word) lady in the middle, but a Vaisya, placing it at the end (of the formula).
\item Let him first beg food of his mother, or of his sister, or of his own maternal aunt, or of (some other) female who will not disgrace him (by a refusal).
\item Having collected as much food as is required (from several persons), and having announced it without guile to his teacher, let him eat, turning his face towards the east, and having purified himself by sipping water.
\item (His meal will procure) long life, if he eats facing the east; fame, if he turns to the south; prosperity, if he turns to the west; truthfulness, if he faces the east.
\item Let a twice-born man always eat his food with concentrated mind, after performing an ablution; and after he has eaten, let him duly cleanse himself with water and sprinkle the cavities (of his head).
\item Let him always worship his food, and eat it without contempt; when he sees it, let him rejoice, show a pleased face, and pray that he may always obtain it.
\item Food, that is always worshipped, gives strength and manly vigour; but eaten irreverently, it destroys them both.
\item Let him not give to any man what he leaves, and beware of eating between (the two meal-times); let him not over-eat himself, nor go anywhere without having purified himself (after his meal).
\item Excessive eating is prejudicial to health, to fame, and to (bliss in) heaven; it prevents (the acquisition of) spiritual merit, and is odious among men; one ought, for these reasons, to avoid it carefully.
\item Let a Brahmana always sip water out of the part of the hand (tirtha) sacred to Brahman, or out of that sacred to Ka (Pragapati), or out of (that) sacred to the gods, never out of that sacred to the manes.
\item They call (the part) at the root of the thumb the tirtha sacred to Brahman, that at the root of the (little) finger (the tirtha) sacred to Ka (Pragapati), (that) at the tips (of the fingers, the tirtha) sacred to the gods, and that below (between the index and the thumb, the tirtha) sacred to the manes.
\item Let him first sip water thrice; next twice wipe his mouth; and, lastly, touch with water the cavities (of the head), (the seat of) the soul and the head.
\item He who knows the sacred law and seeks purity shall always perform the rite of sipping with water neither hot nor frothy, with the (prescribed) tirtha, in a lonely place, and turning to the east or to the north.
\item A Brahmana is purified by water that reaches his heart, a Kshatriya by water reaching his throat, a Vaisya by water taken into his mouth, (and) a Sudra by water touched with the extremity (of his lips).
\item A twice-born man is called upavitin when his right arm is raised (and the sacrificial string or the dress, passed under it, rests on the left shoulder); (when his) left (arm) is raised (and the string, or the dress, passed under it, rests on the right shoulder, he is called) prakinavitin; and nivitin when it hangs down (straight) from the neck.
\item His girdle, the skin (which serves as his upper garment), his staff, his sacrificial thread, (and) his water-pot he must throw into water, when they have been damaged, and take others, reciting sacred formulas.
\item (The ceremony called) Kesanta (clipping the hair) is ordained for a Brahmana in the sixteenth year (from conception); for a Kshatriya, in the twenty-second; and for a Vaisya, two (years) later than that.
\item This whole series (of ceremonies) must be performed for females (also), in order to sanctify the body, at the proper time and in the proper order, but without (the recitation of) sacred texts.
\item The nuptial ceremony is stated to be the Vedic sacrament for women (and to be equal to the initiation), serving the husband (equivalent to) the residence in (the house of the) teacher, and the household duties (the same) as the (daily) worship of the sacred fire.
\item Thus has been described the rule for the initiation of the twice-born, which indicates a (new) birth, and sanctifies; learn (now) to what duties they must afterwards apply themselves.
\item Having performed the (rite of) initiation, the teacher must first instruct the (pupil) in (the rules of) personal purification, of conduct, of the fire-worship, and of the twilight devotions.
\item But (a student) who is about to begin the Study (of the Veda), shall receive instruction, after he has sipped water in accordance with the Institutes (of the sacred law), has made the Brahmangali, (has put on) a clean dress, and has brought his organs under due control.
\item At the beginning and at the end of (a lesson in the) Veda he must always clasp both the feet of his teacher, (and) he must study, joining his hands; that is called the Brahmangali (joining the palms for the sake of the Veda).
\item With crossed hands he must clasp (the feet) of the teacher, and touch the left (foot) with his left (hand), the right (foot) with his right (hand).
\item But to him who is about to begin studying, the teacher always unwearied, must say: Ho, recite! He shall leave off (when the teacher says): Let a stoppage take place!
\item Let him always pronounce the syllable Om at the beginning and at the end of (a lesson in) the Veda; (for) unless the syllable Om precede (the lesson) will slip away (from him), and unless it follow it will fade away.
\item Seated on (blades of Kusa grass) with their points to the east, purified by Pavitras (blades of Kusa grass), and sanctified by three suppressions of the breath (Pranayama), he is worthy (to pronounce) the syllable Om.
\item Pragapati (the lord of creatures) milked out (as it were) from the three Vedas the sounds A, U, and M, and (the Vyahritis) Bhuh, Bhuvah, Svah.
\item Moreover from the three Vedas Pragapati, who dwells in the highest heaven (Parameshthin), milked out (as it were) that Rik-verse, sacred to Savitri (Savitri), which begins with the word tad, one foot from each.
\item A Brahmana, learned in the Veda, who recites during both twilights that syllable and that (verse), preceded by the Vyahritis, gains the (whole) merit which (the recitation of) the Vedas confers.
\item A twice-born man who (daily) repeats those three one thousand times outside (the village), will be freed after a month even from great guilt, as a snake from its slough.
\item The Brahmana, the Kshatriya, and the Vaisya who neglect (the recitation of) that Rik-verse and the timely (performance of the) rites (prescribed for) them, will be blamed among virtuous men.
\item Know that the three imperishable Mahavyahritis, preceded by the syllable Om, and (followed) by the three-footed Savitri are the portal of the Veda and the gate leading (to union with) Brahman.
\item He who daily recites that (verse), untired, during three years, will enter (after death) the highest Brahman, move as free as air, and assume an ethereal form.
\item The monosyllable (Om) is the highest Brahman, (three) suppressions of the breath are the best (form of) austerity, but nothing surpasses the Savitri truthfulness is better than silence.
\item All rites ordained in the Veda, burnt oblations and (other) sacrifices, pass away; but know that the syllable (Om) is imperishable, and (it is) Brahman, (and) the Lord of creatures (Pragapati).
\item An offering, consisting of muttered prayers, is ten times more efficacious than a sacrifice performed according to the rules (of the Veda); a (prayer) which is inaudible (to others) surpasses it a hundred times, and the mental (recitation of sacred texts) a thousand times.
\item The four Pakayagnas and those sacrifices which are enjoined by the rules (of the Veda) are all together not equal in value to a sixteenth part of the sacrifice consisting of muttered prayers.
\item But, undoubtedly, a Brahmana reaches the highest goal by muttering prayers only; (whether) he perform other (rites) or neglect them, he who befriends (all creatures) is declared (to be) a (true) Brahmana.
\item A wise man should strive to restrain his organs which run wild among alluring sensual objects, like a charioteer his horses.
\item Those eleven organs which former sages have named, I will properly (and) precisely enumerate in due order,
\item (Viz.) the ear, the skin, the eyes, the tongue, and the nose as the fifth, the anus, the organ of generation, hands and feet, and the (organ of) speech, named as the tenth.
\item Five of them, the ear and the rest according to their order, they call organs of sense, and five of them, the anus and the rest, organs of action.
\item Know that the internal organ (manas) is the eleventh, which by its quality belongs to both (sets); when that has been subdued, both those sets of five have been conquered.
\item Through the attachment of his organs (to sensual pleasure) a man doubtlessly will incur guilt; but if he keep them under complete control, he will obtain success (in gaining all his aims).
\item Desire is never extinguished by the enjoyment of desired objects; it only grows stronger like a fire (fed) with clarified butter.
\item If one man should obtain all those (sensual enjoyments) and another should renounce them all, the renunciation of all pleasure is far better than the attainment of them.
\item Those (organs) which are strongly attached to sensual pleasures, cannot so effectually be restrained by abstinence (from enjoyments) as by a constant (pursuit of true) knowledge.
\item Neither (the study of) the Vedas, nor liberality, nor sacrifices, nor any (self-imposed) restraint, nor austerities, ever procure the attainment (of rewards) to a man whose heart is contaminated (by sensuality).
\item That man may be considered to have (really) subdued his organs, who on hearing and touching and seeing, on tasting and smelling (anything) neither rejoices nor repines.
\item But when one among all the organs slips away (from control), thereby (man's) wisdom slips away from him, even as the water (flows) through the one (open) foot of a (water-carrier's) skin.
\item If he keeps all the (ten) organs as well as the mind in subjection, he may gain all his aims, without reducing his body by (the practice) of Yoga.
\item Let him stand during the morning twilight, muttering the Savitri until the sun appears, but (let him recite it), seated, in the evening until the constellations can be seen distinctly.
\item He who stands during the morning twilight muttering (the Savitri), removes the guilt contracted during the (previous) night; but he who (recites it), seated, in the evening, destroys the sin he committed during the day.
\item But he who does not (worship) standing in the morning, nor sitting in the evening, shall be excluded, just like a Sudra, from all the duties and rights of an Aryan.
\item He who (desires to) perform the ceremony (of the) daily (recitation), may even recite the Savitri near water, retiring into the forest, controlling his organs and concentrating his mind.
\item Both when (one studies) the supplementary treatises of the Veda, and when (one recites) the daily portion of the Veda, no regard need be paid to forbidden days, likewise when (one repeats) the sacred texts required for a burnt oblation.
\item There are no forbidden days for the daily recitation, since that is declared to be a Brahmasattra (an everlasting sacrifice offered to Brahman); at that the Veda takes the place of the burnt oblations, and it is meritorious (even), when (natural phenomena, requiring) a cessation of the Veda-study, take the place of the exclamation Vashat.
\item For him who, being pure and controlling his organs, during a year daily recites the Veda according to the rule, that (daily recitation) will ever cause sweet and sour milk, clarified butter and honey to flow.
\item Let an Aryan who has been initiated, (daily) offer fuel in the sacred fire, beg food, sleep on the ground and do what is beneficial to this teacher, until (he performs the ceremony of) Samavartana (on returning home).
\item According to the sacred law the (following) ten (persons, viz.) the teacher's son, one who desires to do service, one who imparts knowledge, one who is intent on fulfilling the law, one who is pure, a person connected by marriage or friendship, one who possesses (mental) ability, one who makes presents of money, one who is honest, and a relative, may be instructed (in the Veda).
\item Unless one be asked, one must not explain (anything) to anybody, nor (must one answer) a person who asks improperly; let a wise man, though he knows (the answer), behave among men as (if he were) an idiot.
\item Of the two persons, him who illegally explains (anything), and him who illegally asks (a question), one (or both) will die or incur (the other's) enmity.
\item Where merit and wealth are not (obtained by teaching) nor (at least) due obedience, in such (soil) sacred knowledge must not be sown, just as good seed (must) not (be thrown) on barren land.
\item Even in times of dire distress a teacher of the Veda should rather die with his knowledge than sow it in barren soil.
\item Sacred Learning approached a Brahmana and said to him: `I am thy treasure, preserve me, deliver me not to a scorner; so (preserved) I shall become supremely strong.'
\item `But deliver me, as to the keeper of thy treasure, to a Brahmana whom thou shalt know to be pure, of subdued senses, chaste and attentive.'
\item But he who acquires without permission the Veda from one who recites it, incurs the guilt of stealing the Veda, and shall sink into hell.
\item (A student) shall first reverentially salute that (teacher) from whom he receives (knowledge), referring to worldly affairs, to the Veda, or to the Brahman.
\item A Brahmana who completely governs himself, though he know the Savitri only, is better than he who knows the three Vedas, (but) does not control himself, eats all (sorts of) food, and sells all (sorts of goods).
\item One must not sit down on a couch or seat which a superior occupies; and he who occupies a couch or seat shall rise to meet a (superior), and (afterwards) salute him.
\item For the vital airs of a young man mount upwards to leave his body when an elder approaches; but by rising to meet him and saluting he recovers them.
\item He who habitually salutes and constantly pays reverence to the aged obtains an increase of four (things), (viz.) length of life, knowledge, fame, (and) strength.
\item After the (word of) salutation, a Brahmana who greets an elder must pronounce his name, saying, `I am N. N.'
\item To those (persons) who, when a name is pronounced, do not understand (the meaning of) the salutation, a wise man should say, `It is I;' and (he should address) in the same manner all women.
\item In saluting he should pronounce after his name the word bhoh; for the sages have declared that the nature of bhoh is the same as that of (all proper) names.
\item A Brahmana should thus be saluted in return, `May'st thou be long-lived, O gentle one!' and the vowel `a' must be added at the end of the name (of the person addressed), the syllable preceding it being drawn out to the length of three moras.
\item A Brahmana who does not know the form of returning a salutation, must not be saluted by a learned man; as a Sudra, even so is he.
\item Let him ask a Brahmana, on meeting him, after (his health, with the word) kusala, a Kshatriya (with the word) anamaya, a Vaisya (with the word) kshema, and a Sudra (with the word) anarogya.
\item He who has been initiated (to perform a Srauta sacrifice) must not be addressed by his name, even though he be a younger man; he who knows the sacred law must use in speaking to such (a man the particle) bhoh and (the pronoun) bhavat (your worship).
\item But to a female who is the wife of another man, and not a blood-relation, he must say, `Lady' (bhavati) or `Beloved sister!'
\item To his maternal and paternal uncles, fathers-in-law, officiating priests, (and other) venerable persons, he must say, `I am N. N.,' and rise (to meet them), even though they be younger (than himself).
\item A maternal aunt, the wife of a maternal uncle, a mother-in-law, and a paternal aunt must be honoured like the wife of one's teacher; they are equal to the wife of one's teacher.
\item (The feet of the) wife of one's brother, if she be of the same caste (varna), must be clasped every day; but (the feet of) wives of (other) paternal and maternal relatives need only be embraced on one's return from a journey.
\item Towards a sister of one's father and of one's mother, and towards one's own elder sister, one must behave as towards one's mother; (but) the mother is more venerable than they.
\item Fellow-citizens are called friends (and equals though one be) ten years (older than the other), men practising (the same) fine art (though one be) five years (older than the other), Srotriyas (though) three years (intervene between their ages), but blood-relations only (if the) difference of age be very small.
\item Know that a Brahmana of ten years and Kshatriya of a hundred years stand to each other in the relation of father and son; but between those two the Brahmana is the father.
\item Wealth, kindred, age, (the due performance of) rites, and, fifthly, sacred learning are titles to respect; but each later-named (cause) is more weighty (than the preceding ones).
\item Whatever man of the three (highest) castes possesses most of those five, both in number and degree, that man is worthy of honour among them; and (so is) also a Sudra who has entered the tenth (decade of his life).
\item Way must be made for a man in a carriage, for one who is above ninety years old, for one diseased, for the carrier of a burden, for a woman, for a Snataka, for the king, and for a bridegroom.
\item Among all those, if they meet (at one time), a Snataka and the king must be (most) honoured; and if the king and a Snataka (meet), the latter receives respect from the king.
\item They call that Brahmana who initiates a pupil and teaches him the Veda together with the Kalpa and the Rahasyas, the teacher (akarya, of the latter).
\item But he who for his livelihood teaches a portion only of the Veda, or also the Angas of the Veda, is called the sub-teacher (upadhyaya).
\item That Brahmana, who performs in accordance with the rules (of the Veda) the rites, the Garbhadhana (conception-rite), and so forth, and gives food (to the child), is called the Guru (the venerable one).
\item He who, being (duly) chosen (for the purpose), performs the Agnyadheya, the Pakayagnas, (and) the (Srauta) sacrifices, such as the Agnishtoma (for another man), is called (his) officiating priest.
\item That (man) who truthfully fills both his ears with the Veda, (the pupil) shall consider as his father and mother; he must never offend him.
\item The teacher (akarya) is ten times more venerable than a sub-teacher (upadhyaya), the father a hundred times more than the teacher, but the mother a thousand times more than the father.
\item Of him who gives natural birth and him who gives (the knowledge of) the Veda, the giver of the Veda is the more venerable father; for the birth for the sake of the Veda (ensures) eternal (rewards) both in this (life) and after death.
\item Let him consider that (he received) a (mere animal) existence, when his parents begat him through mutual affection, and when he was born from the womb (of his mother).
\item But that birth which a teacher acquainted with the whole Veda, in accordance with the law, procures for him through the Savitri, is real, exempt from age and death.
\item (The pupil) must know that that man also who benefits him by (instruction in) the Veda, be it little or much, is called in these (Institutes) his Guru, in consequence of that benefit (conferred by instruction in) the Veda.
\item That Brahmana who is the giver of the birth for the sake of the Veda and the teacher of the prescribed duties becomes by law the father of an aged man, even though he himself be a child.
\item Young Kavi, the son of Angiras, taught his (relatives who were old enough to be) fathers, and, as he excelled them in (sacred) knowledge, he called them `Little sons.'
\item They, moved with resentment, asked the gods concerning that matter, and the gods, having assembled, answered, `The child has addressed you properly.'
\item `For (a man) destitute of (sacred) knowledge is indeed a child, and he who teaches him the Veda is his father; for (the sages) have always said ``child'' to an ignorant man, and ``father'' to a teacher of the Veda.'
\item Neither through years, nor through white (hairs), nor through wealth, nor through (powerful) kinsmen (comes greatness). The sages have made this law, `He who has learnt the Veda together with the Angas (Anukana) is (considered) great by us.'
\item The seniority of Brahmanas is from (sacred) knowledge, that of Kshatriyas from valour, that of Vaisyas from wealth in grain (and other goods), but that of Sudras alone from age.
\item A man is not therefore (considered) venerable because his head is gray; him who, though young, has learned the Veda, the gods consider to be venerable.
\item As an elephant made of wood, as an antelope made of leather, such is an unlearned Brahmana; those three have nothing but the names (of their kind).
\item As a eunuch is unproductive with women, as a cow with a cow is unprolific, and as a gift made to an ignorant man yields no reward, even so is a Brahmana useless, who (does) not (know) the Rikas.
\item Created beings must be instructed in (what concerns) their welfare without giving them pain, and sweet and gentle speech must be used by (a teacher) who desires (to abide by) the sacred law.
\item He, forsooth, whose speech and thoughts are pure and ever perfectly guarded, gains the whole reward which is conferred by the Vedanta.
\item Let him not, even though in pain, (speak words) cutting (others) to the quick; let him not injure others in thought or deed; let him not utter speeches which make (others) afraid of him, since that will prevent him from gaining heaven.
\item A Brahmana should always fear homage as if it were poison; and constantly desire (to suffer) scorn as (he would long for) nectar.
\item For he who is scorned (nevertheless may) sleep with an easy mind, awake with an easy mind, and with an easy mind walk here among men; but the scorner utterly perishes.
\item A twice-born man who has been sanctified by the (employment of) the means, (described above) in due order, shall gradually and cumulatively perform the various austerities prescribed for (those who) study the Veda.
\item An Aryan must study the whole Veda together with the Rahasyas, performing at the same time various kinds of austerities and the vows prescribed by the rules (of the Veda).
\item Let a Brahmana who desires to perform austerities, constantly repeat the Veda; for the study of the Veda is declared (to be) in this world the highest austerity for a Brahmana.
\item Verily, that twice-born man performs the highest austerity up to the extremities of his nails, who, though wearing a garland, daily recites the Veda in private to the utmost of his ability.
\item A twice-born man who, not having studied the Veda, applies himself to other (and worldly study), soon falls, even while living, to the condition of a Sudra and his descendants (after him).
\item According to the injunction of the revealed texts the first birth of an Aryan is from (his natural) mother, the second (happens) on the tying of the girdle of Munga grass, and the third on the initiation to (the performance of) a (Srauta) sacrifice.
\item Among those (three) the birth which is symbolised by the investiture with the girdle of Munga grass, is his birth for the sake of the Veda; they declare that in that (birth) the Sivitri (verse) is his mother and the teacher his father.
\item They call the teacher (the pupil's) father because he gives the Veda; for nobody can perform a (sacred) rite before the investiture with the girdle of Munga grass.
\item (He who has not been initiated) should not pronounce (any) Vedic text excepting (those required for) the performance of funeral rites, since he is on a level with a Sudra before his birth from the Veda.
\item The (student) who has been initiated must be instructed in the performance of the vows, and gradually learn the Veda, observing the prescribed rules.
\item Whatever dress of skin, sacred thread, girdle, staff, and lower garment are prescribed for a (student at the initiation), the like (must again be used) at the (performance of the) vows.
\item But a student who resides with his teacher must observe the following restrictive rules, duly controlling all his organs, in order to increase his spiritual merit.
\item Every day, having bathed, and being purified, he must offer libations of water to the gods, sages and manes, worship (the images of) the gods, and place fuel on (the sacred fire).
\item Let him abstain from honey, meat, perfumes, garlands, substances (used for) flavouring (food), women, all substances turned acid, and from doing injury to living creatures.
\item From anointing (his body), applying collyrium to his eyes, from the use of shoes and of an umbrella (or parasol), from (sensual) desire, anger, covetousness, dancing, singing, and playing (musical instruments),
\item From gambling, idle disputes, backbiting, and lying, from looking at and touching women, and from hurting others.
\item Let him always sleep alone, let him never waste his manhood; for he who voluntarily wastes his manhood, breaks his vow.
\item A twice-born student, who has involuntarily wasted his manly strength during sleep, must bathe, worship the sun, and afterwards thrice mutter the Rik-verse (which begins), `Again let my strength return to me.'
\item Let him fetch a pot full of water, flowers, cowdung, earth, and Kusa grass, as much as may be required (by his teacher), and daily go to beg food.
\item A student, being pure, shall daily bring food from the houses of men who are not deficient in (the knowledge of) the Veda and in (performing) sacrifices, and who are famous for (following their lawful) occupations.
\item Let him not beg from the relatives of his teacher, nor from his own or his mother's blood-relations; but if there are no houses belonging to strangers, let him go to one of those named above, taking the last-named first;
\item Or, if there are no (virtuous men of the kind) mentioned above, he may go to each (house in the) village, being pure and remaining silent; but let him avoid Abhisastas (those accused of mortal sin).
\item Having brought sacred fuel from a distance, let him place it anywhere but on the ground, and let him, unwearied, make with it burnt oblations to the sacred fire, both evening and morning.
\item He who, without being sick, neglects during seven (successive) days to go out begging, and to offer fuel in the sacred fire, shall perform the penance of an Avakirnin (one who has broken his vow).
\item He who performs the vow (of studentship) shall constantly subsist on alms, (but) not eat the food of one (person only); the subsistence of a student on begged food is declared to be equal (in merit) to fasting.
\item At his pleasure he may eat, when invited, the food of one man at (a rite) in honour of the gods, observing (however the conditions on his vow, or at a (funeral meal) in honor of the manes, behaving (however) like a hermit.
\item This duty is prescribed by the wise for a Brahmana only; but no such duty is ordained for a Kshatriya and a Vaisya.
\item Both when ordered by his teacher, and without a (special) command, (a student) shall always exert himself in studying (the Veda), and in doing what is serviceable to his teacher.
\item Controlling his body, his speech, his organs (of sense), and his mind, let him stand with joined hands, looking at the face of his teacher.
\item Let him always keep his right arm uncovered, behave decently and keep his body well covered, and when he is addressed (with the words), `Be seated,' he shall sit down, facing his teacher.
\item In the presence of his teacher let him always eat less, wear a less valuable dress and ornaments (than the former), and let him rise earlier (from his bed), and go to rest later.
\item Let him not answer or converse with (his teacher), reclining on a bed, nor sitting, nor eating, nor standing, nor with an averted face.
\item Let him do (that), standing up, if (his teacher) is seated, advancing towards him when he stands, going to meet him if he advances, and running after him when he runs;
\item Going (round) to face (the teacher), if his face is averted, approaching him if he stands at a distance, but bending towards him if he lies on a bed, and if he stands in a lower place.
\item When his teacher is nigh, let his bed or seat be low; but within sight of his teacher he shall not sit carelessly at ease.
\item Let him not pronounce the mere name of his teacher (without adding an honorific title) behind his back even, and let him not mimic his gait, speech, and deportment.
\item Wherever (people) justly censure or falsely defame his teacher, there he must cover his ears or depart thence to another place.
\item By censuring (his teacher), though justly, he will become (in his next birth) an ass, by falsely defaming him, a dog; he who lives on his teacher's substance, will become a worm, and he who is envious (of his merit), a (larger) insect.
\item He must not serve the (teacher by the intervention of another) while he himself stands aloof, nor when he (himself) is angry, nor when a woman is near; if he is seated in a carriage or on a (raised) seat, he must descend and afterwards salute his (teacher).
\item Let him not sit with his teacher, to the leeward or to the windward (of him); nor let him say anything which his teacher cannot hear.
\item He may sit with his teacher in a carriage drawn by oxen, horses, or camels, on a terrace, on a bed of grass or leaves, on a mat, on a rock, on a wooden bench, or in a boat.
\item If his teacher's teacher is near, let him behave (towards him) as towards his own teacher; but let him, unless he has received permission from his teacher, not salute venerable persons of his own (family).
\item This is likewise (ordained as) his constant behaviour towards (other) instructors in science, towards his relatives (to whom honour is due), towards all who may restrain him from sin, or may give him salutary advice.
\item Towards his betters let him always behave as towards his teacher, likewise towards sons of his teacher, born by wives of equal caste, and towards the teacher's relatives both on the side of the father and of the mother.
\item The son of the teacher who imparts instruction (in his father's stead), whether younger or of equal age, or a student of (the science of) sacrifices (or of other Angas), deserves the same honour as the teacher.
\item (A student) must not shampoo the limbs of his teacher's son, nor assist him in bathing, nor eat the fragments of his food, nor wash his feet.
\item The wives of the teacher, who belong to the same caste, must be treated as respectfully as the teacher; but those who belong to a different caste, must be honoured by rising and salutation.
\item Let him not perform for a wife of his teacher (the offices of) anointing her, assisting her in the bath, shampooing her limbs, or arranging her hair.
\item (A pupil) who is full twenty years old, and knows what is becoming and unbecoming, shall not salute a young wife of his teacher (by clasping) her feet.
\item It is the nature of women to seduce men in this (world); for that reason the wise are never unguarded in (the company of) females.
\item For women are able to lead astray in (this) world not only a fool, but even a learned man, and (to make) him a slave of desire and anger.
\item One should not sit in a lonely place with one's mother, sister, or daughter; for the senses are powerful, and master even a learned man.
\item But at his pleasure a young student may prostrate himself on the ground before the young wife of a teacher, in accordance with the rule, and say, `I, N. N., (worship thee, O lady).'
\item On returning from a journey he must clasp the feet of his teacher's wife and daily salute her (in the manner just mentioned), remembering the duty of the virtuous.
\item As the man who digs with a spade (into the ground) obtains water, even so an obedient (pupil) obtains the knowledge which lies (hidden) in his teacher.
\item A (student) may either shave his head, or wear his hair in braids, or braid one lock on the crown of his head; the sun must never set or rise while he (lies asleep) in the village.
\item If the sun should rise or set while he is sleeping, be it (that he offended) intentionally or unintentionally, he shall fast during the (next) day, muttering (the Savitri).
\item For he who lies (sleeping), while the sun sets or rises, and does not perform (that) penance, is tainted by great guilt.
\item Purified by sipping water, he shall daily worship during both twilights with a concentrated mind in a pure place, muttering the prescribed text according to the rule.
\item If a woman or a man of low caste perform anything (leading to) happiness, let him diligently practise it, as well as (any other permitted act) in which his heart finds pleasure.
\item (Some declare that) the chief good consists in (the acquisition of) spiritual merit and wealth, (others place it) in (the gratification of) desire and (the acquisition of) wealth, (others) in (the acquisition of) spiritual merit alone, and (others say that the acquisition of) wealth alone is the chief good here (below); but the (correct) decision is that it consists of the aggregate of (those) three.
\item The teacher, the father, the mother, and an elder brother must not be treated with disrespect, especially by a Brahmana, though one be grievously offended (by them).
\item The teacher is the image of Brahman, the father the image of Pragipati (the lord of created beings), the mother the image of the earth, and an (elder) full brother the image of oneself.
\item That trouble (and pain) which the parents undergo on the birth of (their) children, cannot be compensated even in a hundred years.
\item Let him always do what is agreeable to those (two) and always (what may please) his teacher; when those three are pleased, he obtains all (those rewards which) austerities (yield).
\item Obedience towards those three is declared to be the best (form of) austerity; let him not perform other meritorious acts without their permission.
\item For they are declared to be the three worlds, they the three (principal) orders, they the three Vedas, and they the three sacred fires.
\item The father, forsooth, is stated to be the Garhapatya fire, the mother the Dakshinagni, but the teacher the Ahavaniya fire; this triad of fires is most venerable.
\item He who neglects not those three, (even after he has become) a householder, will conquer the three worlds and, radiant in body like a god, he will enjoy bliss in heaven.
\item By honouring his mother he gains this (nether) world, by honouring his father the middle sphere, but by obedience to his teacher the world of Brahman.
\item All duties have been fulfilled by him who honours those three; but to him who honours them not, all rites remain fruitless.
\item As long as those three live, so long let him not (independently) perform any other (meritorious acts); let him always serve them, rejoicing (to do what is) agreeable and beneficial (to them).
\item He shall inform them of everything that with their consent he may perform in thought, word, or deed for the sake of the next world.
\item By (honouring) these three all that ought to be done by man, is accomplished; that is clearly the highest duty, every other (act) is a subordinate duty.
\item He who possesses faith may receive pure learning even from a man of lower caste, the highest law even from the lowest, and an excellent wife even from a base family.
\item Even from poison nectar may be taken, even from a child good advice, even from a foe (a lesson in) good conduct, and even from an impure (substance) gold.
\item Excellent wives, learning, (the knowledge of) the law, (the rules of) purity, good advice, and various arts may be acquired from anybody.
\item It is prescribed that in times of distress (a student) may learn (the Veda) from one who is not a Brahmana; and that he shall walk behind and serve (such a) teacher, as long as the instruction lasts.
\item He who desires incomparable bliss (in heaven) shall not dwell during his whole life in (the house of) a non-Brahmanical teacher, nor with a Brahmana who does not know the whole Veda and the Angas.
\item But if (a student) desires to pass his whole life in the teacher's house, he must diligently serve him, until he is freed from this body.
\item A Brahmana who serves his teacher till the dissolution of his body, reaches forthwith the eternal mansion of Brahman.
\item He who knows the sacred law must not present any gift to his teacher before (the Samavartana); but when, with the permission of his teacher, he is about to take the (final) bath, let him procure (a present) for the venerable man according to his ability,
\item (Viz.) a field, gold, a cow, a horse, a parasol and shoes, a seat, grain, (even) vegetables, (and thus) give pleasure to his teacher.
\item (A perpetual student) must, if his teacher dies, serve his son (provided he be) endowed with good qualities, or his widow, or his Sapinda, in the same manner as the teacher.
\item Should none of these be alive, he must serve the sacred fire, standing (by day) and sitting (during the night), and thus finish his life.
\item A Brahmana who thus passes his life as a student without breaking his vow, reaches (after death) the highest abode and will not be born again in this world.
\end{enumerate}
