\chapter{}
\begin{enumerate}
\item A twice-born Snataka, who has thus lived according to the law in the order of householders, may, taking a firm resolution and keeping his organs in subjection, dwell in the forest, duly (observing the rules given below).
\item When a householder sees his (skin) wrinkled, and (his hair) white, and. the sons of his sons, then he may resort to the forest.
\item Abandoning all food raised by cultivation, and all his belongings, he may depart into the forest, either committing his wife to his sons, or accompanied by her.
\item Taking with him the sacred fire and the implements required for domestic (sacrifices), he may go forth from the village into the forest and reside there, duly controlling his senses.
\item Let him offer those five great sacrifices according to the rule, with various kinds of pure food fit for ascetics, or with herbs, roots, and fruit.
\item Let him wear a skin or a tattered garment; let him bathe in the evening or in the morning; and let him always wear (his hair in) braids, the hair on his body, his beard, and his nails (being unclipped).
\item Let him perform the Bali-offering with such food as he eats, and give alms according to his ability; let him honour those who come to his hermitage with alms consisting of water, roots, and fruit.
\item Let him be always industrious in privately reciting the Veda; let him be patient of hardships, friendly (towards all), of collected mind, ever liberal and never a receiver of gifts, and compassionate towards all living creatures.
\item Let him offer, according to the law, the Agnihotra with three sacred fires, never omitting the new-moon and full-moon sacrifices at the proper time.
\item Let him also offer the Nakshatreshti, the Agrayana, and the Katurmasya (sacrifices), as well as the Turayana and likewise the Dakshayana, in due order.
\item With pure grains, fit for ascetics, which grow in spring and in autumn, and which he himself has collected, let him severally prepare the sacrificial cakes (purodasa) and the boiled messes (karu), as the law directs.
\item Having offered those most pure sacrificial viands, consisting of the produce of the forest, he may use the remainder for himself, (mixed with) salt prepared by himself.
\item Let him eat vegetables that grow on dry land or in water, flowers, roots, and fruits, the productions of pure trees, and oils extracted from forest-fruits.
\item Let him avoid honey, flesh, and mushrooms growing on the ground (or elsewhere, the vegetables called) Bhustrina, and Sigruka, and the Sleshmantaka fruit.
\item Let him throw away in the month of Asvina the food of ascetics, which he formerly collected, likewise his worn-out clothes and his vegetables, roots, and fruit.
\item Let him not eat anything (grown on) ploughed (land), though it may have been thrown away by somebody, nor roots and fruit grown in a village, though (he may be) tormented (by hunger).
\item He may eat either what has been cooked with fire, or what has been ripened by time; he either may use a stone for grinding, or his teeth may be his mortar.
\item He may either at once (after his daily meal) cleanse (his vessel for collecting food), or lay up a store sufficient for a month, or gather what suffices for six months or for a year.
\item Having collected food according to his ability, he may either eat at night (only), or in the day-time (only), or at every fourth meal-time, or at every eighth.
\item Or he may live according to the rule of the lunar penance (Kandrayana, daily diminishing the quantity of his food) in the bright (half of the month) and (increasing it) in the dark (half); or he may eat on the last days of each fortnight, once (a day only), boiled barley-gruel.
\item Or he may constantly subsist on flowers, roots, and fruit alone, which have been ripened by time and have fallen spontaneously, following the rule of the (Institutes) of Vikhanas.
\item Let him either roll about on the ground, or stand during the day on tiptoe, (or) let him alternately stand and sit down; going at the Savanas (at sunrise, at midday, and at sunset) to water in the forest (in order to bathe).
\item In summer let him expose himself to the heat of five fires, during the rainy season live under the open sky, and in winter be dressed in wet clothes, (thus) gradually increasing (the rigour of) his austerities.
\item When he bathes at the three Savanas (sunrise, midday, and sunset), let him offer libations of water to the manes and the gods, and practising harsher and harsher austerities, let him dry up his bodily frame.
\item Having reposited the three sacred fires in himself, according to the prescribed rule, let him live without a fire, without a house, wholly silent, subsisting on roots and fruit,
\item Making no effort (to procure) things that give pleasure, chaste, sleeping on the bare ground, not caring for any shelter, dwelling at the roots of trees.
\item From Brahmanas (who live as) ascetics, let him receive alms, (barely sufficient) to support life, or from other householders of the twice-born (castes) who reside in the forest.
\item Or (the hermit) who dwells in the forest may bring (food) from a village, receiving it either in a hollow dish (of leaves), in (his naked) hand, or in a broken earthen dish, and may eat eight mouthfuls.
\item These and other observances must a Brahmana who dwells in the forest diligently practise, and in order to attain complete (union with) the (supreme) Soul, (he must study) the various sacred texts contained in the Upanishads,
\item (As well as those rites and texts) which have been practised and studied by the sages (Rishis), and by Brahmana householders, in order to increase their knowledge (of Brahman), and their austerity, and in order to sanctify their bodies;
\item Or let him walk, fully determined and going straight on, in a north-easterly direction, subsisting on water and air, until his body sinks to rest.
\item A Brahmana, having got rid of his body by one of those modes practised by the great sages, is exalted in the world of Brahman, free from sorrow and fear.
\item But having thus passed the third part of (a man's natural term of) life in the forest, he may live as an ascetic during the fourth part of his existence, after abandoning all attachment to worldly objects.
\item He who after passing from order to order, after offering sacrifices and subduing his senses, becomes, tired with (giving) alms and offerings of food, an ascetic, gains bliss after death.
\item When he has paid the three debts, let him apply his mind to (the attainment of) final liberation; he who seeks it without having paid (his debts) sinks downwards.
\item Having studied the Vedas in accordance with the rule, having begat sons according to the sacred law, and having offered sacrifices according to his ability, he may direct his mind to (the attainment of) final liberation.
\item A twice-born man who seeks final liberation, without having studied the Vedas, without having begotten sons, and without having offered sacrifices, sinks downwards.
\item Having performed the Ishti, sacred to the Lord of creatures (Pragapati), where (he gives) all his property as the sacrificial fee, having reposited the sacred fires in himself, a Brahmana may depart from his house (as an ascetic).
\item Worlds, radiant in brilliancy, become (the portion) of him who recites (the texts regarding) Brahman and departs from his house (as an ascetic), after giving a promise of safety to all created beings.
\item For that twice-born man, by whom not the smallest danger even is caused to created beings, there will be no danger from any (quarter), after he is freed from his body.
\item Departing from his house fully provided with the means of purification (Pavitra), let him wander about absolutely silent, and caring nothing for enjoyments that may be offered (to him).
\item Let him always wander alone, without any companion, in order to attain (final liberation), fully understanding that the solitary (man, who) neither forsakes nor is forsaken, gains his end.
\item He shall neither possess a fire, nor a dwelling, he may go to a village for his food, (he shall be) indifferent to everything, firm of purpose, meditating (and) concentrating his mind on Brahman.
\item A potsherd (instead of an alms-bowl), the roots of trees (for a dwelling), coarse worn-out garments, life in solitude and indifference towards everything, are the marks of one who has attained liberation.
\item Let him not desire to die, let him not desire to live; let him wait for (his appointed) time, as a servant (waits) for the payment of his wages.
\item Let him put down his foot purified by his sight, let him drink water purified by (straining with) a cloth, let him utter speech purified by truth, let him keep his heart pure.
\item Let him patiently bear hard words, let him not insult anybody, and let him not become anybody's enemy for the sake of this (perishable) body.
\item Against an angry man let him not in return show anger, let him bless when he is cursed, and let him not utter speech, devoid of truth, scattered at the seven gates.
\item Delighting in what refers to the Soul, sitting (in the postures prescribed by the Yoga), independent (of external help), entirely abstaining from sensual enjoyments, with himself for his only companion, he shall live in this world, desiring the bliss (of final liberation).
\item Neither by (explaining) prodigies and omens, nor by skill in astrology and palmistry, nor by giving advice and by the exposition (of the Sastras), let him ever seek to obtain alms.
\item Let him not (in order to beg) go near a house filled with hermits, Brahmanas, birds, dogs, or other mendicants.
\item His hair, nails, and beard being clipped, carrying an alms-bowl, a staff, and a water-pot, let him continually wander about, controlling himself and not hurting any creature.
\item His vessels shall not be made of metal, they shall be free from fractures; it is ordained that they shall be cleansed with water, like (the cups, called) Kamasa, at a sacrifice.
\item A gourd, a wooden bowl, an earthen (dish), or one made of split cane, Manu, the son of Svayambhu, has declared (to be) vessels (suitable) for an ascetic.
\item Let him go to beg once (a day), let him not be eager to obtain a large quantity (of alms); for an ascetic who eagerly seeks alms, attaches himself also to sensual enjoyments.
\item When no smoke ascends from (the kitchen), when the pestle lies motionless, when the embers have been extinguished, when the people have finished their meal, when the remnants in the dishes have been removed, let the ascetic always go to beg.
\item Let him not be sorry when he obtains nothing, nor rejoice when he obtains (something), let him (accept) so much only as will sustain life, let him not care about the (quality of his) utensils.
\item Let him disdain all (food) obtained in consequence of humble salutations, (for) even an ascetic who has attained final liberation, is bound (with the fetters of the Samsara) by accepting (food given) in consequence of humble salutations.
\item By eating little, and by standing and sitting in solitude, let him restrain his senses, if they are attracted by sensual objects.
\item By the restraint of his senses, by the destruction of love and hatred, and by the abstention from injuring the creatures, he becomes fit for immortality.
\item Let him reflect on the transmigrations of men, caused by their sinful deeds, on their falling into hell, and on the torments in the world of Yama,
\item On the separation from their dear ones, on their union with hated men, on their being overpowered by age and being tormented with diseases,
\item On the departure of the individual soul from this body and its new birth in (another) womb, and on its wanderings through ten thousand millions of existences,
\item On the infliction of pain on embodied (spirits), which is caused by demerit, and the gain of eternal bliss, which is caused by the attainment of their highest aim, (gained through) spiritual merit.
\item By deep meditation let him recognise the subtile nature of the supreme Soul, and its presence in all organisms, both the highest and the lowest.
\item To whatever order he may be attached, let him, though blemished (by a want of the external marks), fulfil his duty, equal-minded towards all creatures; (for) the external mark (of the order) is not the cause of (the acquisition of) merit.
\item Though the fruit of the Kataka tree (the clearing-nut) makes water clear, yet the (latter) does not become limpid in consequence of the mention of the (fruit's) name.
\item In order to preserve living creatures, let him always by day and by night, even with pain to his body, walk, carefully scanning the ground.
\item In order to expiate (the death) of those creatures which he unintentionally injures by day or by night, an ascetic shall bathe and perform six suppressions of the breath.
\item Three suppressions of the breath even, performed according to the rule, and accompanied with the (recitation of the) Vyahritis and of the syllable Om, one must know to be the highest (form of) austerity for every Brahmana.
\item For as the impurities of metallic ores, melted in the blast (of a furnace), are consumed, even so the taints of the organs are destroyed through the suppression of the breath.
\item Let him destroy the taints through suppressions of the breath, (the production of) sin by fixed attention, all sensual attachments by restraining (his senses and organs), and all qualities that are not lordly by meditation.
\item Let him recognise by the practice of meditation the progress of the individual soul through beings of various kinds, (a progress) hard to understand for unregenerate men.
\item He who possesses the true insight (into the nature of the world), is not fettered by his deeds; but he who is destitute of that insight, is drawn into the circle of births and deaths.
\item By not injuring any creatures, by detaching the senses (from objects of enjoyment), by the rites prescribed in the Veda, and by rigorously practising austerities, (men) gain that state (even) in this (world).
\item Let him quit this dwelling, composed of the five elements, where the bones are the beams, which is held together by tendons (instead of cords), where the flesh and the blood are the mortar, which is thatched with the skin, which is foul-smelling, filled with urine and ordure, infested by old age and sorrow, the seat of disease, harassed by pain, gloomy with passion, and perishable.
\item He who leaves this body, (be it by necessity) as a tree (that is torn from) the river-bank, or (freely) like a bird (that) quits a tree, is freed from the misery (of this world, dreadful like) a shark.
\item Making over (the merit of his own) good actions to his friends and (the guilt of) his evil deeds to his enemies, he attains the eternal Brahman by the practice of meditation.
\item When by the disposition (of his heart) he becomes indifferent to all objects, he obtains eternal happiness both in this world and after death.
\item He who has in this manner gradually given up all attachments and is freed from all the pairs (of opposites), reposes in Brahman alone.
\item All that has been declared (above) depends on meditation; for he who is not proficient in the knowledge of that which refers to the Soul reaps not the full reward of the performance of rites.
\item Let him constantly recite (those texts of) the Veda which refer to the sacrifice, (those) referring to the deities, and (those) which treat of the Soul and are contained in the concluding portions of the Veda (Vedanta).
\item That is the refuge of the ignorant, and even that (the refuse) of those who know (the meaning of the Veda); that is (the protection) of those who seek (bliss in) heaven and of those who seek endless (beatitude).
\item A twice-born man who becomes an ascetic, after the successive performance of the above-mentioned acts, shakes off sin here below and reaches the highest Brahman.
\item Thus the law (valid) for self-restrained ascetics has been explained to you; now listen to the (particular) duties of those who give up (the rites prescribed by) the Veda.
\item The student, the householder, the hermit, and the ascetic, these (constitute) four separate orders, which all spring from (the order of) householders.
\item But all (or) even (any of) these orders, assumed successively in accordance with the Institutes (of the sacred law), lead the Brahmana who acts by the preceding (rules) to the highest state.
\item And in accordance with the precepts of the Veda and of the Smriti, the housekeeper is declared to be superior to all of them; for he supports the other three.
\item As all rivers, both great and small, find a resting-place in the ocean, even so men of all orders find protection with householders
\item By twice-born men belonging to (any of) these four orders, the tenfold law must be ever carefully obeyed.
\item Contentment, forgiveness, self-control, abstention from unrighteously appropriating anything, (obedience to the rules of) purification, coercion of the organs, wisdom, knowledge (of the supreme Soul), truthfulness, and abstention from anger, (form) the tenfold law.
\item Those Brahmanas who thoroughly study the tenfold law, and after studying obey it, enter the highest state.
\item A twice-born man who, with collected mind, follows the tenfold law and has paid his (three) debts, may, after learning the Vedanta according to the prescribed rule, become an ascetic.
\item Having given up (the performance of) all rites, throwing off the guilt of his (sinful) acts, subduing his organs and having studied the Veda, he may live at his ease under the protection of his son.
\item He who has thus given up (the performance of) all rites, who is solely intent on his own (particular) object, (and) free from desires, destroys his guilt by his renunciation and obtains the highest state.
\item Thus the fourfold holy law of Brahmanas, which after death (yields) imperishable rewards, has been declared to you; now learn the duty of kings.
\end{enumerate}
